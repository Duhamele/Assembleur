\documentclass[10pt,a4paper]{article}

\usepackage[utf8]{inputenc}
\usepackage[T1]{fontenc}
\usepackage[french]{babel}
\usepackage{amsmath}
\usepackage{listings}
\usepackage{hyperref}
\title{NASM X86}
\begin{document}
	\maketitle
	\section{Resgistre}
	\begin{tabular}{|c|c|c|c|c|c|}
		\hline
	Regitre 64b	&registre 32b  & resgistre 16b  &resgistre 8b  & resgistre 8b 15-8 &  descrisptive\\
		\hline
	RAX/R0	& EAX/R0D  & AX/R0W & AL/R0B &AH  &  \\
		\hline
	RCX/R1	& ECX/R1D  & CX/R1W & CL/R1B  &CH  &  \\
		\hline
	RDX/R2	& EDX/R2D &DX/R2W  & DL/R2B & DH &  \\
		\hline
	RBX/R3	& EBX/R3D  &BX/R3W  & BL/R3B & BH &  \\
		\hline
	RSP/R4	& ESP/R4D  & SP/R4W  & SP/R4B &  & Adresse Pile  \\
		\hline
	RBP/R5	& EBP/R5D & BP/R5W  & BPL/R5B &  &  \\
		\hline
	RSI/R6	&ESI/R6D  & SI/R6W & SIL/R6B &  &  \\
		\hline
	RDI/R7	& EDI/R7D  & DI/R7W & DIL/R7B  &  &  \\
		\hline
	R8-R15	& R8D-R15D & R8W-R15W & R8B-R15B  &  &  \\
		\hline
	\end{tabular}
	Registre Xmm, 128bit pour les vecteurs.\\
	Xmm0-Xmm15


\section{Argument}
	les six premiers argument sont:\\
	(rdi, rsi, rdx, rcx, r8, r9).
	
\section{Division}
En assembleur x86-64 (NASM), les opérations arithmétiques de multiplication et de division sont réalisées avec des instructions spécifiques :
\begin{itemize}
	\item \textbf{mul} : multiplication non signée.
	\item \textbf{imul} : multiplication signée.
	\item \textbf{div} : division non signée.
	\item \textbf{idiv} : division signée.
\end{itemize}

Ces instructions utilisent des registres prédéfinis, en particulier \texttt{RAX} et \texttt{RDX}.

\section{Instruction \texttt{mul} (non signée)}
\subsection{Description}
\texttt{mul op} réalise : \texttt{RAX = RAX * op}. Le résultat complet est stocké dans le couple \texttt{RDX:RAX}.

\begin{center}
	\begin{tabular}{|c|c|c|}
		\hline
		Taille opérande & Multiplication & Résultat \\
		\hline
		8 bits & AL × op & AX \\
		16 bits & AX × op & DX:AX \\
		32 bits & EAX × op & EDX:EAX \\
		64 bits & RAX × op & RDX:RAX \\
		\hline
	\end{tabular}
\end{center}

\subsection{Exemple}
\begin{lstlisting}[language={[x86masm]Assembler}]
	mov     rax, 6
	mov     rbx, 7
	mul     rbx        ; RAX = 42, RDX = 0
\end{lstlisting}

\section{Instruction \texttt{imul} (signée)}
\subsection{Modes d'utilisation}
\begin{itemize}
	\item \textbf{1 opérande :} \texttt{imul op} → \texttt{RAX = RAX * op}
	\item \textbf{2 opérandes :} \texttt{imul reg, op} → \texttt{reg = reg * op}
	\item \textbf{3 opérandes :} \texttt{imul reg, op1, op2} → \texttt{reg = op1 * op2}
\end{itemize}

\subsection{Exemple}
\begin{lstlisting}[language={[x86masm]Assembler}]
	mov     rax, -5
	mov     rbx, 3
	imul    rbx        ; RAX = -15
\end{lstlisting}

\section{Division : \texttt{div} et \texttt{idiv}}
\subsection{Description}
Avant la division, le dividende doit être placé dans le couple \texttt{RDX:RAX}.

\begin{center}
	\begin{tabular}{|c|c|c|c|}
		\hline
		Taille & Dividende & Quotient & Reste \\
		\hline
		8 bits & AX & AL & AH \\
		16 bits & DX:AX & AX & DX \\
		32 bits & EDX:EAX & EAX & EDX \\
		64 bits & RDX:RAX & RAX & RDX \\
		\hline
	\end{tabular}
\end{center}

\subsection{Exemple non signé}
\begin{lstlisting}[language={[x86masm]Assembler}]
	mov     rax, 50      ; dividende bas
	xor     rdx, rdx     ; RDX = 0 (dividende haut)
	mov     rbx, 7       ; diviseur
	div     rbx          ; RAX = 7, RDX = 1
\end{lstlisting}

\subsection{Exemple signé (\texttt{idiv})}
\begin{lstlisting}[language={[x86masm]Assembler}]
	mov     rax, -50
	cqo                 ; étend le signe dans RDX (convert quadword)
	mov     rbx, 7
	idiv    rbx         ; RAX = -7, RDX = -1
\end{lstlisting}

\section{Résumé}
\begin{itemize}
	\item \texttt{mul} et \texttt{div} : nombres non signés.
	\item \texttt{imul} et \texttt{idiv} : nombres signés.
	\item Résultat des multiplications longues dans \texttt{RDX:RAX}.
	\item Pour diviser, le dividende doit être dans \texttt{RDX:RAX}.
\end{itemize}

\end{document}